\documentclass[11pt]{article}

\usepackage[russian]{babel}
\usepackage[T2A]{fontenc}
\usepackage[utf8]{inputenc}
\usepackage{amsthm}
\usepackage{dsfont}
\usepackage{kpfonts}
\usepackage[
paperwidth = 148 mm,
paperheight = 210 mm,
left = 1 cm,
right = 1 cm,
top = 1 cm,
bottom = 1 cm
]{geometry}
\begin{document}
\section{\underline{Центральная предельная теорема}}
    \newtheorem{thm}{Теорема}[section]
    \begin{thm}[Линдеберга]
    	Пусть $\{\xi_k\}_(k>=1)$~--- независимая случайная величина, $E \xi_k^2 < +\infty \forall k$, обозначим $m_k = E \xi_k$, $\sigma_k^2 = D\xi_k > 0$; $S_n = \sum_{i = 1}^{n} \xi_i$; $D^2_n = \sum_{k = 1}^{n} \sigma_k^2$ и $F_k(x)$~--- функция распределения $\xi_k$. Пусть выполнено условие Линдеберга, т.е.
    	$$
    	\forall \varepsilon > 0 \frac{1}{D_n^2} \sum_{k = 1}^{n} \int\limits_{\{x:|x-m_k| > \varepsilon D_n \}} (x-m_k)^2 d~F_k(x) \xrightarrow[\mathsf{n} \rightarrow \infty] 0
    	$$.
    	Тогда $\frac{S_n - E S_n}{\sqrt{DS_n}} \overset{d}{\longrightarrow} \infty $.
    \end{thm}
\section{\underline{Гауссовские случайные векторы}}
\newtheorem{definition}[section]{Определение}
\begin{definition}
Случайный вектор $\overrightarrow{\xi} ~ N()$~--- гаусс., если его характеристическая функция

$$
\varphi\xi(\vec{t}) = exp(i(\vec{m},\vec{t}) - \frac{1}{2}(\sum \vec{t},\vec{t})), \vec{m}\in\mathds{R}^n, \sum
$$
~--- сумма.
\end{definition}
\end{document}
